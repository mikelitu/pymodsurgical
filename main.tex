%%%%%%%%%%%%%%%%%%%%%%%%%%%%%%%%%%%%%%%%%%%%%%%%%%%%%%%%%%%%%%%
%
% Welcome to Overleaf --- just edit your LaTeX on the left,
% and we'll compile it for you on the right. If you open the
% 'Share' menu, you can invite other users to edit at the same
% time. See www.overleaf.com/learn for more info. Enjoy!
%
%%%%%%%%%%%%%%%%%%%%%%%%%%%%%%%%%%%%%%%%%%%%%%%%%%%%%%%%%%%%%%%
\documentclass{article}
\usepackage{amsmath}
\title{Amsmath example}
\author{Overleaf}
\date{\today}
\begin{document}
\maketitle
\section{Introduction}

Dynamics under small deformations are well approximated by a system of masses, dampers and spring.

\begin{equation*}
    \mathbf{M\ddot{u}}(t) + \mathbf{C\dot{u}}(t) + \mathbf{Ku}(t) = \mathbf{f}(t)
\end{equation*}

where $\mathbf{\ddot{u}}(t)$, $\mathbf{\dot{u}}(t)$, and $\mathbf{u}(t)$ are vectors for acceleration, velocity and displacement. If we assume that the equation has a sinusoidal solution, the eigenmodes are orthogonal and given by $\mathbf{K} \phi_{i} = \omega^{2}_{i} \mathbf{M} \phi_{i}$. The set of eigenvectors $\phi_{1} \text{...} \phi_{N}$.

\begin{equation*}
    \mathbf{\Phi} = \left[\phi_{1} \quad \phi_{2} \quad \text{ ... } \quad \phi_{N}\right]
\end{equation*}
\begin{equation*}
    \mathbf{\Phi}^{T}\mathbf{M\Phi} = diag(\mathbf{m}_{i})
\end{equation*}
\begin{equation*}
    \mathbf{\Phi}^{T}\mathbf{K\Phi} = diag(\mathbf{k}_{i})
\end{equation*}

The matrix $\mathbf{\Phi}$ defines modal coordinates $\mathbf{q}(t)$ where $\mathbf{u}(t) = \mathbf{\Phi q}(t)$.
\begin{equation*}
    \mathbf{\ddot{q}}(t) + 2\xi_{i}\omega_{i}\mathbf{\dot{q}}(t) + \omega_{i}^{2}\mathbf{q}(t) = \frac{\mathbf{f_{i}}}{\mathbf{m_{i}}}
\end{equation*}

where the undamped natural frequency is $\omega_{i} = \sqrt{\frac{\mathbf{k_{i}}}{m_{i}}}$, giving the modal damping factor.

\begin{equation*}
    \xi_{i} = \frac{\mathbf{c_{i}}}{2\mathbf{m_{i}}\omega_{i}} = \frac{1}{2}\left(\frac{\alpha}{\omega_{i}} + \beta\omega_{i}\right)
\end{equation*}

We can obtain the unit impulse response for the $i^{th}$.

\begin{equation*}
    h_{i}(t) = \left(\frac{e^{-\xi_{i}\omega_{i}t}}{\mathbf{m_{i}}\omega_{di}}\right) sin{(\omega_{di}t)}
\end{equation*}

where the damped natural frequency is $\omega_{di} = \omega_{i} \sqrt{1 - \xi^{2}_{i}}$. Taking the Fourier transform of the unit impulse response $h_{i}(t)$.

\begin{equation*}
    H_{i}(\omega) = \left(\frac{1}{}\right)
\end{equation*}

\end{document}